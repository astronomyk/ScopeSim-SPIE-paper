Possible conferences preferences
--------------------------------
1. Ground-based and Airborne Instrumentation for Astronomy VIII
https://spie.org/AS/conferencedetails/astronomy-ground-based-instrumentation

2. Modeling, Systems Engineering, and Project Management for Astronomy IX
https://spie.org/AS/conferencedetails/astronomy-modeling-systems-engineering

3. Ground-based and Airborne Telescopes VIII
https://spie.org/AS/conferencedetails/ground-air-telescopes

Abstract for online - 1000 characters
-------------------------------------

ScopeSim is a python-based software framework that allows the user to generate
simulated images of astronomical objects as would be seen by the detectors of
an arbitrary optical system. The package allows multiple types of optical system
to be modelled: from simple imaging cameras to integral field unit spectrographs.

The software architecture is based on a physically motivated model common to all
astronomical optical systems. The model mirrors the main components of any real
optical system: a spectro-spatial distribution of light passing through a series
of optical sub-systems (atmosphere, telescope, fore-optics, instrument) before
being converted into photo-electrons in a detector.

ScopeSim itself is however instrument-agnostic. When in use, a ScopeSim
instrument model rerequired data from an ``instrument package'', which are
downloaded as needed from an external server. Currently such packages have been
created for two of the ELT's first-light instruments: MICADO and METIS.


Abstract for technical review - 20000 characters
------------------------------------------------

What is ScopeSim

Scope of simulations
- Full raw data products for pipelines
- Reduced stacked data for science feasibility studies
- Restricted mode advanced ETC functionality

How does ScopeSim work?
- brief description
- 4 planes
- 3 control classes
- effects objects
- simulation run

Where does ScopeSim get its data from?
- IRDB
- MICADO, METIS, HAWKI
- Other packages


What is ScopeSim?
-----------------
ScopeSim is a pythonic instrument data simulation framework, which allows the
user to generate output data from an astronomical optical (telescope +
instrument) system in a variety of formats. The package has been designed and
written in such a way that is is capable of modelling any system that follows
the standard astronomical optical train: a spectro-spatial distribution of light
passing through a series of optical sub-systems (atmosphere, telescope,
fore-optics, instrument) before being converted into photo-electrons in a
detector.

The software itself is agnostic to the optical train it must model, and
relies on the information provided by so-called ``instrument packages''.
These contain the descriptions of any optical aberations or effects incurred
along the optical train, the data and/or values associated with these
effects, and, in special cases, also the software implementation of any
custom effects. The online ScopeSim documenation provides a detailed
description of the data formats and interfaces that need to be observed when
constructing an instrument package.

The raison d'etre for ScopeSim is to fill the niche that exists between the
two major families of observation simulations currently pursued by instrument
consortia: the exposure time calculator (ETC) and the end-to-end (E2E ) model.
ETCs provide the most basic information about the observability of a target,
generally be referencing pre-calculated look-up tables, and are therefore
well suited as general-purpose support tools for those interested in the
capabilities of an instrument. E2E models are at the other extreme, and
often recompute the paths of photons or light-rays through a detailed model
of an optical system every time a simulation is run. The very high level of
accuracy achieved by E2E simulations comes at the expense of compuational
efficiency. Such simulation are indispensible for understanding the fine
details of true optical characterises of an instrument, however they are
impractical for rapid turn-around applications such as refining an
observation strategy or determining the extent to which a bright field star
will impact the outcome of a proposal.

Given the ever increasing pressure for telescope time as well as the
increasing complexity of instrumentation systems, there is a very real need
for a set of observation simulation tools which allow astronomers to go
beyond the single-value metric of an ETC, but does not require the
infrastructure and time-investment associated with E2E simulations. In recent
years several consortia have developed solutions to fill this niche, driven
primarily by the needs of the in-house science team and the astronomical
community. A prime example is the software suite developed by the JWST team
[!cite JWST!]. Unfortunately though there is still no generic software
package which allows astronomers to compare observations with different
instruments on a single coherent framework. It is exactly this niche that
ScopeSim aims to fill.


Scope of simulations
--------------------
- Full raw data products for pipelines
- Reduced stacked data for science feasibility studies
- Restricted mode advanced ETC functionality


How does ScopeSim work?
-----------------------
ScopeSim consists of two major software components, and a database system. The
software components are split between the tasks of describing an on-sky object
and observing this object. The database contains the data necessary to model the
effects of the optical system on the incoming light. These three components are
referred to as the ScopeSim engine (scopesim), the ScopeSim templates library
(scopesim_templates) and the instrument reference database (IRDB). Each
component will be described in detail below.

The ScopeSim engine
+++++++++++++++++++
The main workhorse class of the ScopeSim engine is the OpticalTrain class. This
class is responsible for accepting a Source object from the templates library,
extracting the relevant field of view, generating a model of the optical system
in memory from data supplied by the IRDB, applying all optical and electronic
aberrations and effects, and finally piecing together the output in the format
defined by the user (FITS images, spectra, etc).

An OpticalTrain uses a physically motivated series of actions to observe an
on-sky object. ScopeSim does not use either ray-tracing or
Fraunhofen/Frenel-optics methods to create images of the target. Instead it
cuts the target description into many spatial and spectral slices, applies the
necessary effects to each slice, and projects each slice onto a focal plane
canvas, known as the ``photo-electron expectation map''. Here electronic
effects are applied and the Detector objects extract the information needed to
produce the final data products. The effects used are each their own Python
class and can contain either analytical or numerical descriptions of the action.
While the ScopeSim engine package already contains a large list of common
optical effects, custom effects can be added by third parties using the
IRDB effect plug-in framework.

Internally the OpticalTrain contains two manager classes: the field-of-view
manager (FOVManager) and the effects manager (OpticsManager). The FOVManager
maintains an overview of which sections of the spectro-spatial domain has been
sampled and projected onto the focal plane. The OpticsManager collates and
controls when and how each optical effect is applied to the incoming flux
slices. For example, it would be computationally inefficient to apply the
transmission curves of each optical surface individually to the flux slices, so
the OpticsManager generates a system transmission curve when the OpticalTrain is
initialised and applies only this single curve to spectra used to describe the
target.

This architecture gives ScopeSim great flexibility. By splitting the Source
object into quasi-monochromatic image chunks, ScopeSim mimics what happens
inside a real optical system. Each chunk is shifted, distorted, blurred, and
extincted as if it were passing through the various optical surfaces
of a real observatory. Due to this method, most variations on the classic
optical system can be implemented in ScopeSim with relative ease. A long slit
spectrograph is an imager where the monochromatic slices are dispersed
over the focal plane. An integral field spectrograph can be modelled as a series
of long slit spectrographs. A fibre-fed multi-object spectrograph is a series of
hexagonal ``slits'' which throw away spatial information. A coronagraph is an
imager with a customised PSF description. Each of these systems can be
implemented relatively painlessly with the ScopeSim architecture as long as an
analytical or (interpolatable) numerical description exists for each
instrumental effect in the form of a ScopeSim Effect object.

Not only does this software architecture allow most types of instruments to be
implemented in ScopeSim, it also means that astronomers will be able to use a
single platform to compare simulated observations in an apples-to-apples
fashion.

Effects already implemented in the ScopeSim engine include:
- Spectral effects:
    - Mirror transmission
    - Mirror greybody emission
    - Customisable sky transmission and emission, downloaded directly from the
      ESO SkyCalc tool API

- Spatial effects:
    - Seeing and diffraction limited PSF, both varying and constant in the
      spatial and spectral domains
    - Spectrograph trace maps
    - Atmospheric diffraction
    - Non common path aberrations
    - Wind shake and vibrations
    - Optical distorsion

- Electronic effects
    - Quantum efficiency
    - Dark current
    - Various form of read noise (mainly for HAWAII HgCdTe detectors)
    - Linearity and saturation
    - Shot noise

This is not an exhaustive list of effects, and more will be added over time as
the package matures. The community is also invited to contribute to this list
via Pull-Requests on the ScopeSim GitHub page.


The ScopeSim template library
+++++++++++++++++++++++++++++






- functions to create Source objects
- can be made from




IRDB
++++
Th


- yaml
- plug-in effects
- hosting
- build your own





By using this method ScopeSim is able to
both keep memory and processor requirements to a minimum, while









- series of interlinking parts
    - Description of source (ScopeSim_templates)
    - Information about optical train and observing modes (IRDB packages)
    - Physical model
    - Effects
    - Controlling simulations










- brief description
- 4 planes
- 3 control classes
- effects objects
- simulation run



























