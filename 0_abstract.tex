Possible conferences preferences
--------------------------------
1. Ground-based and Airborne Instrumentation for Astronomy VIII
https://spie.org/AS/conferencedetails/astronomy-ground-based-instrumentation

2. Modeling, Systems Engineering, and Project Management for Astronomy IX
https://spie.org/AS/conferencedetails/astronomy-modeling-systems-engineering

3. Ground-based and Airborne Telescopes VIII
https://spie.org/AS/conferencedetails/ground-air-telescopes

Abstract for online - 1000 characters
-------------------------------------

ScopeSim is a python-based software framework that allows the user to generate
simulated images of astronomical objects as would be seen by the detectors of
an arbitrary optical system. The package allows multiple types of optical system
to be modelled: from simple imaging cameras to integral field unit spectrographs.

The software architecture is based on a physically motivated model common to all
astronomical optical systems. The model mirrors the main components of any real
optical system: a spectro-spatial distribution of light passing through a series
of optical sub-systems (atmosphere, telescope, fore-optics, instrument) before
being converted into photo-electrons in a detector.

ScopeSim itself is however instrument-agnostic. When in use, a ScopeSim
instrument model rerequired data from an ``instrument package'', which are
downloaded as needed from an external server. Currently such packages have been
created for two of the ELT's first-light instruments: MICADO and METIS.


Abstract for technical review - 20000 characters
------------------------------------------------

What is ScopeSim

Scope of simulations
- Full raw data products for pipelines
- Reduced stacked data for science feasibility studies
- Restricted mode advanced ETC functionality

How does ScopeSim work?
- brief description
- 4 planes
- 3 control classes
- effects objects
- simulation run

Where does ScopeSim get its data from?
- IRDB
- MICADO, METIS, HAWKI
- Other packages


What is ScopeSim?
ScopeSim is a pythonic instrument data simulation framework, which allows the
user to generate output data from an astronomical optical (telescope +
instrument) system in a variety of formats. The package has been designed and
written in such a way that is is capable of modelling any system that follows
the standard astronomical optical train: a spectro-spatial distribution of light
passing through a series of optical sub-systems (atmosphere, telescope,
fore-optics, instrument) before being converted into photo-electrons in a
detector.

The software itself is agnostic to the optical train it must model, and
relies on the information provided by so-called ``instrument packages''.
These contain the descriptions of any optical aberations or effects incurred
along the optical train, the data and/or values associated with these
effects, and, in special cases, also the software implementation of any
custom effects. The online ScopeSim documenation provides a detailed
description of the data formats and interfaces that need to be observed when
constructing an instrument package.

The raison d'etre for ScopeSim is to fill the niche that exists between the
two major families of observation simulations currently pursued by instrument
consortia: the exposure time calculator (ETC) and the end-to-end (E2E ) model.
ETCs provide the most basic information about the observability of a target,
generally be referencing pre-calculated look-up tables, and are therefore
well suited as general-purpose support tools for those interested in the
capabilities of an instrument. E2E models are at the other extreme, and
often recompute the paths of photons or light-rays through a detailed model
of an optical system every time a simulation is run. The very high level of
accuracy achieved by E2E simulations comes at the expense of compuational
efficiency. Such simulation are indispensible for understanding the fine
details of true optical characterises of an instrument, however they are
impractical for rapid turn-around applications such as refining an
observation strategy or determining the extent to which a bright field star
will impact the outcome of a proposal.

Given the ever increasing pressure for telescope time as well as the
increasing complexity of instrumentation systems, there is a very real need
for a set of observation simulation tools which allow astronomers to go
beyond the single-value metric of an ETC, but does not require the
infrastructure and time-investment associated with E2E simulations. In recent
years several consortia have developed solutions to fill this niche, driven
primarily by the needs of the in-house science team and the astronomical
community. A prime example is the software suite developed by the JWST team
[!cite JWST!]. Unfortunately though there is still no generic software
package which allows astronomers to compare observations with different
instruments on a single coherent framework. It is exactly this niche that
ScopeSim aims to fill.


Scope of simulations
- Full raw data products for pipelines
- Reduced stacked data for science feasibility studies
- Restricted mode advanced ETC functionality






























