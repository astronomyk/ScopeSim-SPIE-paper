Possible conferences preferences
--------------------------------
1. Ground-based and Airborne Instrumentation for Astronomy VIII
https://spie.org/AS/conferencedetails/astronomy-ground-based-instrumentation

2. Modeling, Systems Engineering, and Project Management for Astronomy IX
https://spie.org/AS/conferencedetails/astronomy-modeling-systems-engineering

3. Ground-based and Airborne Telescopes VIII
https://spie.org/AS/conferencedetails/ground-air-telescopes

Abstract for online - 1000 characters
-------------------------------------

ScopeSim is a python-based software framework that allows the user to generate
simulated images of astronomical objects as would be seen by the detectors of
an arbitrary optical system. The package allows multiple types of optical system
to be modelled: from simple imaging cameras to integral field unit spectrographs.

The software architecture is based on a physically motivated model common to all
astronomical optical systems. The model mirrors the main components of any real
optical system: a spectro-spatial distribution of light passing through a series
of optical sub-systems (atmosphere, telescope, fore-optics, instrument) before
being converted into photo-electrons in a detector.

ScopeSim itself is however instrument-agnostic. When in use, a ScopeSim
instrument model rerequired data from an ``instrument package'', which are
downloaded as needed from an external server. Currently such packages have been
created for two of the ELT's first-light instruments: MICADO and METIS.


Abstract for technical review - 20000 characters
------------------------------------------------

What is ScopeSim

Scope of simulations
- Full raw data products for pipelines
- Reduced stacked data for science feasibility studies
- Restricted mode advanced ETC functionality

How does ScopeSim work?
- brief description
- 4 planes
- 3 control classes
- effects objects
- simulation run

Where does ScopeSim get its data from?
- IRDB
- MICADO, METIS, HAWKI
- Other packages





































